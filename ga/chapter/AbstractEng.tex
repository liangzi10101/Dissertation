\abstract{Performance bottleneck identification technology is essential for the performance optimization algorithm for complex rendering system. With the continuous development of games and other markets, users' demand for high-quality rendering pictures is increasing day by day. Rendering systems of digital games and virtual reality applications often need to add more rendering algorithms to meet users' needs, but more rendering algorithms means greater consumption of computing resources and time, which will directly affect the performance of real-time rendering. On the one hand, the optimization of the algorithm implementation process itself becomes a focus. But on the other hand, even if the algorithm itself achieves the optimization, it may still become a bad factor affecting the real-time rendering performance in a specific rendering scene, resulting in rendering not smoothly. Therefore, performance bottleneck identification technology has become an important means to help developers further optimize the rendering solution.

In this paper, we propose an algorithm to identify the local performance bottlenecks on algorithm level for a complex rendering system, which is based on several complex rendering algorithms and arbitrarily designated dynamic scenes. The purpose of the algorithm is to help developers find out the local algorithm level bottlenecks that affects the real-time rendering performance while ensuring the rendering results. Based on the previous work of building bottleneck analysis model by machine learning, this paper proposes an algorithm level performance bottleneck identification method based on partition of feature space, which improves the performance identification instability and solves the problem of non-global optimal solution. The main contents of this paper are as follows:

First, refine and construct a rendering performance evaluation model based on regression forest algorithm using pre-collecting performance data sets, and further improve the realization process of variable importance as algorithm bottleneck measurement standard.

Second, a quantitative evaluation criterion for performance bottleneck tree is proposed and designed on the premise of performance bottleneck tree solution based on feature space partitioning data set for the situation that the global performance bottleneck cannot be found in rendering performance data sets.

Third, the partition task is converted to a global optimization problem and two new methods are employed to solve this problem, including greedy strategy and quasi-genetic strategy. In order to make the quasi-genetic strategy applicable to our problem, we propose to use a pseudo full binary tree to represent the process and result of a partition. It improves the problem of morphological instability of performance bottleneck tree and non-global optimal solution in previous research.

Fourth, in order to guide the improvement of performance bottleneck tree construction process, further validation work has been done on the phenomenon of sensitivity to pre-collected data volume in the process of variable importance calculation.

Finally, Our experimental results suggest that our method can find a reasonable partition to split the original performance data set into several subsets in which bottlenecks can be identified.

\textbf{Performance bottleneck analysis; Bottleneck analysis tree; Genetic algorithm; Greedy Algorithm}}
